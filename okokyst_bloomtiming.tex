% Options for packages loaded elsewhere
\PassOptionsToPackage{unicode}{hyperref}
\PassOptionsToPackage{hyphens}{url}
%
\documentclass[
]{article}
\usepackage{amsmath,amssymb}
\usepackage{lmodern}
\usepackage{ifxetex,ifluatex}
\ifnum 0\ifxetex 1\fi\ifluatex 1\fi=0 % if pdftex
  \usepackage[T1]{fontenc}
  \usepackage[utf8]{inputenc}
  \usepackage{textcomp} % provide euro and other symbols
\else % if luatex or xetex
  \usepackage{unicode-math}
  \defaultfontfeatures{Scale=MatchLowercase}
  \defaultfontfeatures[\rmfamily]{Ligatures=TeX,Scale=1}
\fi
% Use upquote if available, for straight quotes in verbatim environments
\IfFileExists{upquote.sty}{\usepackage{upquote}}{}
\IfFileExists{microtype.sty}{% use microtype if available
  \usepackage[]{microtype}
  \UseMicrotypeSet[protrusion]{basicmath} % disable protrusion for tt fonts
}{}
\makeatletter
\@ifundefined{KOMAClassName}{% if non-KOMA class
  \IfFileExists{parskip.sty}{%
    \usepackage{parskip}
  }{% else
    \setlength{\parindent}{0pt}
    \setlength{\parskip}{6pt plus 2pt minus 1pt}}
}{% if KOMA class
  \KOMAoptions{parskip=half}}
\makeatother
\usepackage{xcolor}
\IfFileExists{xurl.sty}{\usepackage{xurl}}{} % add URL line breaks if available
\IfFileExists{bookmark.sty}{\usepackage{bookmark}}{\usepackage{hyperref}}
\hypersetup{
  pdftitle={Time of first bloom of major phytoplankton groups along the Norwegian coast},
  pdfauthor={EEG},
  hidelinks,
  pdfcreator={LaTeX via pandoc}}
\urlstyle{same} % disable monospaced font for URLs
\usepackage[margin=1in]{geometry}
\usepackage{color}
\usepackage{fancyvrb}
\newcommand{\VerbBar}{|}
\newcommand{\VERB}{\Verb[commandchars=\\\{\}]}
\DefineVerbatimEnvironment{Highlighting}{Verbatim}{commandchars=\\\{\}}
% Add ',fontsize=\small' for more characters per line
\usepackage{framed}
\definecolor{shadecolor}{RGB}{248,248,248}
\newenvironment{Shaded}{\begin{snugshade}}{\end{snugshade}}
\newcommand{\AlertTok}[1]{\textcolor[rgb]{0.94,0.16,0.16}{#1}}
\newcommand{\AnnotationTok}[1]{\textcolor[rgb]{0.56,0.35,0.01}{\textbf{\textit{#1}}}}
\newcommand{\AttributeTok}[1]{\textcolor[rgb]{0.77,0.63,0.00}{#1}}
\newcommand{\BaseNTok}[1]{\textcolor[rgb]{0.00,0.00,0.81}{#1}}
\newcommand{\BuiltInTok}[1]{#1}
\newcommand{\CharTok}[1]{\textcolor[rgb]{0.31,0.60,0.02}{#1}}
\newcommand{\CommentTok}[1]{\textcolor[rgb]{0.56,0.35,0.01}{\textit{#1}}}
\newcommand{\CommentVarTok}[1]{\textcolor[rgb]{0.56,0.35,0.01}{\textbf{\textit{#1}}}}
\newcommand{\ConstantTok}[1]{\textcolor[rgb]{0.00,0.00,0.00}{#1}}
\newcommand{\ControlFlowTok}[1]{\textcolor[rgb]{0.13,0.29,0.53}{\textbf{#1}}}
\newcommand{\DataTypeTok}[1]{\textcolor[rgb]{0.13,0.29,0.53}{#1}}
\newcommand{\DecValTok}[1]{\textcolor[rgb]{0.00,0.00,0.81}{#1}}
\newcommand{\DocumentationTok}[1]{\textcolor[rgb]{0.56,0.35,0.01}{\textbf{\textit{#1}}}}
\newcommand{\ErrorTok}[1]{\textcolor[rgb]{0.64,0.00,0.00}{\textbf{#1}}}
\newcommand{\ExtensionTok}[1]{#1}
\newcommand{\FloatTok}[1]{\textcolor[rgb]{0.00,0.00,0.81}{#1}}
\newcommand{\FunctionTok}[1]{\textcolor[rgb]{0.00,0.00,0.00}{#1}}
\newcommand{\ImportTok}[1]{#1}
\newcommand{\InformationTok}[1]{\textcolor[rgb]{0.56,0.35,0.01}{\textbf{\textit{#1}}}}
\newcommand{\KeywordTok}[1]{\textcolor[rgb]{0.13,0.29,0.53}{\textbf{#1}}}
\newcommand{\NormalTok}[1]{#1}
\newcommand{\OperatorTok}[1]{\textcolor[rgb]{0.81,0.36,0.00}{\textbf{#1}}}
\newcommand{\OtherTok}[1]{\textcolor[rgb]{0.56,0.35,0.01}{#1}}
\newcommand{\PreprocessorTok}[1]{\textcolor[rgb]{0.56,0.35,0.01}{\textit{#1}}}
\newcommand{\RegionMarkerTok}[1]{#1}
\newcommand{\SpecialCharTok}[1]{\textcolor[rgb]{0.00,0.00,0.00}{#1}}
\newcommand{\SpecialStringTok}[1]{\textcolor[rgb]{0.31,0.60,0.02}{#1}}
\newcommand{\StringTok}[1]{\textcolor[rgb]{0.31,0.60,0.02}{#1}}
\newcommand{\VariableTok}[1]{\textcolor[rgb]{0.00,0.00,0.00}{#1}}
\newcommand{\VerbatimStringTok}[1]{\textcolor[rgb]{0.31,0.60,0.02}{#1}}
\newcommand{\WarningTok}[1]{\textcolor[rgb]{0.56,0.35,0.01}{\textbf{\textit{#1}}}}
\usepackage{graphicx}
\makeatletter
\def\maxwidth{\ifdim\Gin@nat@width>\linewidth\linewidth\else\Gin@nat@width\fi}
\def\maxheight{\ifdim\Gin@nat@height>\textheight\textheight\else\Gin@nat@height\fi}
\makeatother
% Scale images if necessary, so that they will not overflow the page
% margins by default, and it is still possible to overwrite the defaults
% using explicit options in \includegraphics[width, height, ...]{}
\setkeys{Gin}{width=\maxwidth,height=\maxheight,keepaspectratio}
% Set default figure placement to htbp
\makeatletter
\def\fps@figure{htbp}
\makeatother
\setlength{\emergencystretch}{3em} % prevent overfull lines
\providecommand{\tightlist}{%
  \setlength{\itemsep}{0pt}\setlength{\parskip}{0pt}}
\setcounter{secnumdepth}{-\maxdimen} % remove section numbering
\ifluatex
  \usepackage{selnolig}  % disable illegal ligatures
\fi

\title{Time of first bloom of major phytoplankton groups along the
Norwegian coast}
\author{EEG}
\date{27 9 2021}

\begin{document}
\maketitle

The aim of these analyses is to look at patterns in the timing of the
first bloom event of the year, between and within the major
phytoplankton groups Bacillariophyceae (Diatoms) and Dinophyceae
(Dinoflagellates), at the Økokyst stations, ordered from north to south.
This may reveal temporal or spatial niche separation. A `bloom event' is
here defined as registration of cell carbon \textgreater{} 85 microg C
m\^{}-3, which is the bloom level defined according to Lømsland \&
Johnsen (2016). Other metrics/parameters may also be plotted.

\hypertarget{read-data-files}{%
\subsubsection{Read data files}\label{read-data-files}}

\hypertarget{file-of-stations-with-latitude-and-longitude-i-had-problems-with-the-utm-coordinates}{%
\paragraph{File of stations with latitude and longitude (I had problems
with the UTM
coordinates)}\label{file-of-stations-with-latitude-and-longitude-i-had-problems-with-the-utm-coordinates}}

NB. Lat and long for the stations starting with SBR are missing, need to
find them, or convert UTM coordinates properly.

\begin{Shaded}
\begin{Highlighting}[]
\NormalTok{stations }\OtherTok{\textless{}{-}} \FunctionTok{read\_xlsx}\NormalTok{(}\FunctionTok{here}\NormalTok{(}\StringTok{"data"}\NormalTok{, }\StringTok{"Copy of OKOKYST\_Hydrografi\_Stasjoner\_v5.xlsx"}\NormalTok{), }\AttributeTok{sheet =} \StringTok{"KML"}\NormalTok{)}
\end{Highlighting}
\end{Shaded}

\hypertarget{read-data-file-of-cell-carbon}{%
\paragraph{Read data file of cell
carbon}\label{read-data-file-of-cell-carbon}}

Downloaded from the `Vannmiljø'-site of Miljødirektoratet.

\begin{Shaded}
\begin{Highlighting}[]
\NormalTok{cellcarbon0 }\OtherTok{\textless{}{-}} \FunctionTok{read\_xlsx}\NormalTok{(}\FunctionTok{here}\NormalTok{(}\StringTok{"data"}\NormalTok{, }\StringTok{"okokyst\_cellcarbon.xlsx"}\NormalTok{)) }

\NormalTok{cellcarbon1 }\OtherTok{\textless{}{-}}\NormalTok{ cellcarbon0 }\SpecialCharTok{\%\textgreater{}\%}\NormalTok{ tidyr}\SpecialCharTok{::}\FunctionTok{separate}\NormalTok{(Tid\_provetak, }\FunctionTok{c}\NormalTok{(}\StringTok{"Year"}\NormalTok{, }\StringTok{"Month"}\NormalTok{, }\StringTok{"Day"}\NormalTok{), }\StringTok{"{-}"}\NormalTok{, }\AttributeTok{remove =} \ConstantTok{FALSE}\NormalTok{) }\SpecialCharTok{\%\textgreater{}\%} 
\NormalTok{  tidyr}\SpecialCharTok{::}\FunctionTok{separate}\NormalTok{(Day, }\FunctionTok{c}\NormalTok{(}\StringTok{"Day"}\NormalTok{, }\StringTok{"Time"}\NormalTok{), }\StringTok{" "}\NormalTok{, }\AttributeTok{remove =}\NormalTok{ T) }\SpecialCharTok{\%\textgreater{}\%} \FunctionTok{mutate}\NormalTok{(}\AttributeTok{verdi =} \FunctionTok{as.numeric}\NormalTok{(Verdi)) }\SpecialCharTok{\%\textgreater{}\%} 
    \FunctionTok{mutate}\NormalTok{(}\FunctionTok{across}\NormalTok{(Tid\_provetak, }\SpecialCharTok{\textasciitilde{}} \FunctionTok{as.Date}\NormalTok{(}\FunctionTok{as.character}\NormalTok{(.), }\AttributeTok{format =} \StringTok{\textquotesingle{}\%Y{-}\%m{-}\%d\textquotesingle{}}\NormalTok{)))}
\end{Highlighting}
\end{Shaded}

\hypertarget{read-data-file-of-taxonomy-of-the-speciescategories-join-with-cell-carbon-file}{%
\paragraph{Read data file of taxonomy of the species/categories, join
with cell carbon
file}\label{read-data-file-of-taxonomy-of-the-speciescategories-join-with-cell-carbon-file}}

The taxonomy file has been mostly prepared by Sandra. Will need some
further edits, but for this purpose, i.e.~grouping taxa according to
genus or class, it should be fine.

\begin{Shaded}
\begin{Highlighting}[]
\NormalTok{tax }\OtherTok{\textless{}{-}} \FunctionTok{read\_delim}\NormalTok{(}\FunctionTok{here}\NormalTok{(}\StringTok{"data"}\NormalTok{, }\StringTok{"okokyst\_taxonomy.txt"}\NormalTok{), }\AttributeTok{delim =} \StringTok{"}\SpecialCharTok{\textbackslash{}t}\StringTok{"}\NormalTok{)}
\NormalTok{cellcarbon\_tax }\OtherTok{\textless{}{-}} \FunctionTok{left\_join}\NormalTok{(cellcarbon1, tax, }\AttributeTok{by =} \FunctionTok{c}\NormalTok{(}\StringTok{"name"}\NormalTok{)) }\SpecialCharTok{\%\textgreater{}\%} 
  \FunctionTok{mutate}\NormalTok{(}\AttributeTok{doy =}\NormalTok{ lubridate}\SpecialCharTok{::}\FunctionTok{yday}\NormalTok{(Tid\_provetak)) }\SpecialCharTok{\%\textgreater{}\%} \CommentTok{\#Add variable day{-}of{-}year}
  \FunctionTok{left\_join}\NormalTok{(., stations, }\AttributeTok{by =} \FunctionTok{c}\NormalTok{(}\StringTok{"station\_code"} \OtherTok{=} \StringTok{"StationCode"}\NormalTok{)) }\SpecialCharTok{\%\textgreater{}\%} \CommentTok{\# Join with station file with Latitude and Longitude}
\NormalTok{  dplyr}\SpecialCharTok{::}\FunctionTok{arrange}\NormalTok{(., }\AttributeTok{by =}\NormalTok{ Latitude)}
\end{Highlighting}
\end{Shaded}

\hypertarget{define-bloom-level.-in-the-table-of-cell-carbon-from-vannmiljuxf8-cell-carbon-is-given-in-units-of-microgc-l-1.}{%
\paragraph{Define bloom level. In the table of cell carbon from
Vannmiljø cell carbon is given in units of microgC
L\^{}-1.}\label{define-bloom-level.-in-the-table-of-cell-carbon-from-vannmiljuxf8-cell-carbon-is-given-in-units-of-microgc-l-1.}}

\begin{Shaded}
\begin{Highlighting}[]
\NormalTok{bloomlevel }\OtherTok{\textless{}{-}} \FloatTok{0.085} \CommentTok{\#ygC/L, cf. Lømsland \& Johnsen}
\end{Highlighting}
\end{Shaded}

\hypertarget{compare-time-of-first-bloom-in-diatoms-and-dinoflagellates-for-each-station-ordered-from-north-to-south}{%
\subsubsection{Compare time of first bloom in diatoms and
dinoflagellates, for each station (ordered from north to
south)}\label{compare-time-of-first-bloom-in-diatoms-and-dinoflagellates-for-each-station-ordered-from-north-to-south}}

\begin{Shaded}
\begin{Highlighting}[]
\NormalTok{level1 }\OtherTok{\textless{}{-}} \StringTok{"Class"}
\NormalTok{level2 }\OtherTok{\textless{}{-}} \StringTok{"Class"}


\NormalTok{firstbloom\_diat }\OtherTok{\textless{}{-}}\NormalTok{ cellcarbon\_tax }\SpecialCharTok{\%\textgreater{}\%} \FunctionTok{mutate}\NormalTok{(}\AttributeTok{bloom =} \FunctionTok{ifelse}\NormalTok{(Verdi }\SpecialCharTok{\textgreater{}}\NormalTok{ bloomlevel, }\DecValTok{1}\NormalTok{, }\DecValTok{0}\NormalTok{)) }\SpecialCharTok{\%\textgreater{}\%} 
  \FunctionTok{filter}\NormalTok{(.data[[level1]] }\SpecialCharTok{\%in\%} \FunctionTok{c}\NormalTok{(}\StringTok{"Bacillariophyta"}\NormalTok{)) }\SpecialCharTok{\%\textgreater{}\%} \FunctionTok{filter}\NormalTok{(bloom }\SpecialCharTok{==} \DecValTok{1}\NormalTok{) }\SpecialCharTok{\%\textgreater{}\%} 
  \FunctionTok{group\_by}\NormalTok{(.data[[level2]], name, Year, station\_code, StationName\_new) }\SpecialCharTok{\%\textgreater{}\%} \FunctionTok{summarise\_at}\NormalTok{(}\FunctionTok{vars}\NormalTok{(doy), min) }\SpecialCharTok{\%\textgreater{}\%}  \CommentTok{\# add name when level2 != name}
  \FunctionTok{mutate}\NormalTok{(}\AttributeTok{Station\_code =} \FunctionTok{factor}\NormalTok{(station\_code, }\AttributeTok{levels =} \FunctionTok{unique}\NormalTok{(cellcarbon\_tax}\SpecialCharTok{$}\NormalTok{station\_code), }\AttributeTok{ordered =}\NormalTok{ T))}

\NormalTok{firstbloom\_dino }\OtherTok{\textless{}{-}}\NormalTok{ cellcarbon\_tax }\SpecialCharTok{\%\textgreater{}\%} \FunctionTok{mutate}\NormalTok{(}\AttributeTok{bloom =} \FunctionTok{ifelse}\NormalTok{(Verdi }\SpecialCharTok{\textgreater{}}\NormalTok{ bloomlevel, }\DecValTok{1}\NormalTok{, }\DecValTok{0}\NormalTok{)) }\SpecialCharTok{\%\textgreater{}\%} 
  \FunctionTok{filter}\NormalTok{(.data[[level1]] }\SpecialCharTok{\%in\%} \FunctionTok{c}\NormalTok{(}\StringTok{"Dinophyceae"}\NormalTok{)) }\SpecialCharTok{\%\textgreater{}\%} \FunctionTok{filter}\NormalTok{(bloom }\SpecialCharTok{==} \DecValTok{1}\NormalTok{) }\SpecialCharTok{\%\textgreater{}\%} 
  \FunctionTok{group\_by}\NormalTok{(.data[[level2]], name, Year, station\_code, StationName\_new) }\SpecialCharTok{\%\textgreater{}\%} \FunctionTok{summarise\_at}\NormalTok{(}\FunctionTok{vars}\NormalTok{(doy), min) }\SpecialCharTok{\%\textgreater{}\%}  \CommentTok{\# add name when level2 != name}
  \FunctionTok{mutate}\NormalTok{(}\AttributeTok{Station\_code =} \FunctionTok{factor}\NormalTok{(station\_code, }\AttributeTok{levels =} \FunctionTok{unique}\NormalTok{(cellcarbon\_tax}\SpecialCharTok{$}\NormalTok{station\_code), }\AttributeTok{ordered =}\NormalTok{ T))}

\NormalTok{firstbloom\_both }\OtherTok{\textless{}{-}}\NormalTok{ cellcarbon\_tax }\SpecialCharTok{\%\textgreater{}\%} \FunctionTok{mutate}\NormalTok{(}\AttributeTok{bloom =} \FunctionTok{ifelse}\NormalTok{(Verdi }\SpecialCharTok{\textgreater{}}\NormalTok{ bloomlevel, }\DecValTok{1}\NormalTok{, }\DecValTok{0}\NormalTok{)) }\SpecialCharTok{\%\textgreater{}\%} 
  \FunctionTok{filter}\NormalTok{(.data[[level1]] }\SpecialCharTok{\%in\%} \FunctionTok{c}\NormalTok{(}\StringTok{"Bacillariophyta"}\NormalTok{, }\StringTok{"Dinophyceae"}\NormalTok{)) }\SpecialCharTok{\%\textgreater{}\%} \FunctionTok{filter}\NormalTok{(bloom }\SpecialCharTok{==} \DecValTok{1}\NormalTok{) }\SpecialCharTok{\%\textgreater{}\%} 
  \FunctionTok{group\_by}\NormalTok{(.data[[level2]], name, Year, station\_code, StationName\_new) }\SpecialCharTok{\%\textgreater{}\%} \FunctionTok{summarise\_at}\NormalTok{(}\FunctionTok{vars}\NormalTok{(doy), min) }\SpecialCharTok{\%\textgreater{}\%}  \CommentTok{\# add name when level2 != name}
  \FunctionTok{mutate}\NormalTok{(}\AttributeTok{Station\_code =} \FunctionTok{factor}\NormalTok{(station\_code, }\AttributeTok{levels =} \FunctionTok{unique}\NormalTok{(cellcarbon\_tax}\SpecialCharTok{$}\NormalTok{station\_code), }\AttributeTok{ordered =}\NormalTok{ T))}
\end{Highlighting}
\end{Shaded}

\begin{Shaded}
\begin{Highlighting}[]
\NormalTok{cbPalette }\OtherTok{\textless{}{-}} \FunctionTok{c}\NormalTok{(}\StringTok{"\#999999"}\NormalTok{, }\StringTok{"\#E69F00"}\NormalTok{, }\StringTok{"\#56B4E9"}\NormalTok{, }\StringTok{"\#009E73"}\NormalTok{, }\StringTok{"\#F0E442"}\NormalTok{, }\StringTok{"\#0072B2"}\NormalTok{, }\StringTok{"\#D55E00"}\NormalTok{, }\StringTok{"\#CC79A7"}\NormalTok{)}

\NormalTok{firstbloom\_diat\_plt }\OtherTok{\textless{}{-}} \FunctionTok{ggplot}\NormalTok{(firstbloom\_diat, }\FunctionTok{aes}\NormalTok{(}\AttributeTok{y =}\NormalTok{ Station\_code, }\AttributeTok{x =}\NormalTok{ doy, }\AttributeTok{color =}\NormalTok{ .data[[level2]], }\AttributeTok{shape =}\NormalTok{ Year)) }\SpecialCharTok{+}
  \FunctionTok{geom\_jitter}\NormalTok{(}\AttributeTok{alpha =} \DecValTok{1}\SpecialCharTok{/}\DecValTok{2}\NormalTok{)}\SpecialCharTok{+}
  \FunctionTok{scale\_colour\_manual}\NormalTok{(}\AttributeTok{values =}\NormalTok{ cbPalette[}\DecValTok{3}\NormalTok{])}

\NormalTok{firstbloom\_diat\_plt}
\end{Highlighting}
\end{Shaded}

\begin{figure}
\centering
\includegraphics{okokyst_bloomtiming_files/figure-latex/unnamed-chunk-6-1.pdf}
\caption{First registration of the year of cell carbon above `bloom
level' for each species/taxon of diatom registered in the Okokyst data.}
\end{figure}

\begin{Shaded}
\begin{Highlighting}[]
\NormalTok{firstbloom\_dino\_plt }\OtherTok{\textless{}{-}} \FunctionTok{ggplot}\NormalTok{(firstbloom\_dino, }\FunctionTok{aes}\NormalTok{(}\AttributeTok{x =}\NormalTok{ Station\_code, }\AttributeTok{y =}\NormalTok{ doy, }\AttributeTok{color =}\NormalTok{ .data[[level2]], }\AttributeTok{shape =}\NormalTok{ Year, }
                       \AttributeTok{text =} \FunctionTok{sprintf}\NormalTok{(}\StringTok{"Taxon: \%s\textless{}br\textgreater{}Station: \%s\textless{}br\textgreater{} Doy: \%s"}\NormalTok{, .data[[level2]], StationName\_new, doy))) }\SpecialCharTok{+}
  \FunctionTok{geom\_jitter}\NormalTok{(}\AttributeTok{alpha =} \DecValTok{1}\SpecialCharTok{/}\DecValTok{2}\NormalTok{)}\SpecialCharTok{+}
  \FunctionTok{coord\_flip}\NormalTok{()}\SpecialCharTok{+}
  \FunctionTok{scale\_colour\_manual}\NormalTok{(}\AttributeTok{values =}\NormalTok{ cbPalette[}\DecValTok{2}\NormalTok{])}

\NormalTok{firstbloom\_dino\_plt}
\end{Highlighting}
\end{Shaded}

\begin{figure}
\centering
\includegraphics{okokyst_bloomtiming_files/figure-latex/unnamed-chunk-7-1.pdf}
\caption{First registration of the year of cell carbon above `bloom
level' for each species/taxon of dinoflagellate registered in the
Okokyst data.}
\end{figure}

\begin{Shaded}
\begin{Highlighting}[]
\NormalTok{firstbloom\_plt }\OtherTok{\textless{}{-}} \FunctionTok{ggplot}\NormalTok{(firstbloom\_both, }\FunctionTok{aes}\NormalTok{(}\AttributeTok{x =}\NormalTok{ Station\_code, }\AttributeTok{y =}\NormalTok{ doy, }\AttributeTok{color =}\NormalTok{ .data[[level2]], }\AttributeTok{shape =}\NormalTok{ Year, }
                       \AttributeTok{text =} \FunctionTok{sprintf}\NormalTok{(}\StringTok{"Taxon: \%s\textless{}br\textgreater{}Station: \%s\textless{}br\textgreater{} Doy: \%s"}\NormalTok{, .data[[level2]], StationName\_new, doy))) }\SpecialCharTok{+}
  \FunctionTok{geom\_jitter}\NormalTok{(}\AttributeTok{alpha =} \DecValTok{1}\SpecialCharTok{/}\DecValTok{2}\NormalTok{, }\AttributeTok{height =} \FloatTok{1.5}\NormalTok{, }\AttributeTok{width =} \FloatTok{1.5}\NormalTok{)}\SpecialCharTok{+}
  \FunctionTok{coord\_flip}\NormalTok{()}\SpecialCharTok{+}
  \FunctionTok{scale\_colour\_manual}\NormalTok{(}\AttributeTok{values =}\NormalTok{ cbPalette[}\FunctionTok{c}\NormalTok{(}\DecValTok{3}\NormalTok{,}\DecValTok{2}\NormalTok{)])}

\NormalTok{firstbloom\_plt}
\end{Highlighting}
\end{Shaded}

\includegraphics{okokyst_bloomtiming_files/figure-latex/unnamed-chunk-8-1.pdf}
\#\#\# Ridge plots

\begin{Shaded}
\begin{Highlighting}[]
\NormalTok{for\_ridges }\OtherTok{\textless{}{-}}\NormalTok{ firstbloom\_both }\SpecialCharTok{\%\textgreater{}\%} \FunctionTok{group\_by}\NormalTok{(Class, Station\_code, doy) }\SpecialCharTok{\%\textgreater{}\%} \FunctionTok{count}\NormalTok{() }\SpecialCharTok{\%\textgreater{}\%} 
  \FunctionTok{mutate}\NormalTok{(}\AttributeTok{Station\_code =} \FunctionTok{factor}\NormalTok{(Station\_code, }\AttributeTok{levels =} \FunctionTok{rev}\NormalTok{(}\FunctionTok{unique}\NormalTok{(cellcarbon\_tax}\SpecialCharTok{$}\NormalTok{station\_code)), }\AttributeTok{ordered =}\NormalTok{ T))}
  
\FunctionTok{ggplot}\NormalTok{(for\_ridges, }\FunctionTok{aes}\NormalTok{(}\AttributeTok{y =}\NormalTok{ Class, }\AttributeTok{x =}\NormalTok{ doy, }\AttributeTok{fill =}\NormalTok{ Class))}\SpecialCharTok{+}
\FunctionTok{geom\_density\_ridges}\NormalTok{(}\AttributeTok{stat =} \StringTok{"binline"}\NormalTok{, }\AttributeTok{scale =} \DecValTok{1}\NormalTok{, }\AttributeTok{binwidth =} \DecValTok{5}\NormalTok{)}\SpecialCharTok{+}
  \FunctionTok{xlim}\NormalTok{(}\DecValTok{0}\NormalTok{, }\DecValTok{366}\NormalTok{)}\SpecialCharTok{+}
  \FunctionTok{facet\_grid}\NormalTok{(}\AttributeTok{rows =} \FunctionTok{vars}\NormalTok{(Station\_code))}
\end{Highlighting}
\end{Shaded}

\includegraphics{okokyst_bloomtiming_files/figure-latex/fig 3-1.pdf}
Conclusions:~ - Diatoms vs.~Dinoflagellates: Overlap between bloom
timings of diatoms and dinoflagellates.\\
- North - South gradient: Tendency for a higher number of species
blooming earlier (as early as January) with decreasing latitude.

\hypertarget{compare-time-of-first-bloom-of-abundant-genera-within-the-diatoms}{%
\subsubsection{Compare time of first bloom of abundant genera within the
diatoms}\label{compare-time-of-first-bloom-of-abundant-genera-within-the-diatoms}}

\begin{Shaded}
\begin{Highlighting}[]
\NormalTok{level1 }\OtherTok{\textless{}{-}} \StringTok{"Genus"}
\NormalTok{level2 }\OtherTok{\textless{}{-}} \StringTok{"Genus"}


\NormalTok{firstbloom\_comp\_diat }\OtherTok{\textless{}{-}}\NormalTok{ cellcarbon\_tax }\SpecialCharTok{\%\textgreater{}\%} \FunctionTok{mutate}\NormalTok{(}\AttributeTok{bloom =} \FunctionTok{ifelse}\NormalTok{(Verdi }\SpecialCharTok{\textgreater{}}\NormalTok{ bloomlevel, }\DecValTok{1}\NormalTok{, }\DecValTok{0}\NormalTok{)) }\SpecialCharTok{\%\textgreater{}\%} 
  \FunctionTok{filter}\NormalTok{(.data[[level1]] }\SpecialCharTok{\%in\%} \FunctionTok{c}\NormalTok{(}\StringTok{"Skeletonema"}\NormalTok{, }\StringTok{"Chaetoceros"}\NormalTok{, }\StringTok{"Pseudo{-}nitzschia"}\NormalTok{)) }\SpecialCharTok{\%\textgreater{}\%} \FunctionTok{filter}\NormalTok{(bloom }\SpecialCharTok{==} \DecValTok{1}\NormalTok{) }\SpecialCharTok{\%\textgreater{}\%} 
  \FunctionTok{group\_by}\NormalTok{(.data[[level2]], name, Year, station\_code, StationName\_new) }\SpecialCharTok{\%\textgreater{}\%} \FunctionTok{summarise\_at}\NormalTok{(}\FunctionTok{vars}\NormalTok{(doy), min) }\SpecialCharTok{\%\textgreater{}\%}  \CommentTok{\# add name when level2 != name}
  \FunctionTok{mutate}\NormalTok{(}\AttributeTok{Station\_code =} \FunctionTok{factor}\NormalTok{(station\_code, }\AttributeTok{levels =} \FunctionTok{unique}\NormalTok{(cellcarbon\_tax}\SpecialCharTok{$}\NormalTok{station\_code), }\AttributeTok{ordered =}\NormalTok{ T))}
\end{Highlighting}
\end{Shaded}

\begin{Shaded}
\begin{Highlighting}[]
\NormalTok{firstbloom\_comp\_diat\_plot }\OtherTok{\textless{}{-}} \FunctionTok{ggplot}\NormalTok{(firstbloom\_comp\_diat, }\FunctionTok{aes}\NormalTok{(}\AttributeTok{x =}\NormalTok{ Station\_code, }\AttributeTok{y =}\NormalTok{ doy, }\AttributeTok{color =}\NormalTok{ .data[[level2]], }\AttributeTok{shape =}\NormalTok{ Year, }
                       \AttributeTok{text =} \FunctionTok{sprintf}\NormalTok{(}\StringTok{"Taxon: \%s\textless{}br\textgreater{}Station: \%s\textless{}br\textgreater{} Doy: \%s"}\NormalTok{, .data[[level2]], StationName\_new, doy))) }\SpecialCharTok{+}
  \FunctionTok{geom\_jitter}\NormalTok{(}\AttributeTok{alpha =} \DecValTok{1}\SpecialCharTok{/}\DecValTok{2}\NormalTok{, }\AttributeTok{height =} \FloatTok{1.5}\NormalTok{, }\AttributeTok{width =} \FloatTok{1.5}\NormalTok{)}\SpecialCharTok{+}
  \FunctionTok{coord\_flip}\NormalTok{()}\SpecialCharTok{+}
  \FunctionTok{scale\_colour\_manual}\NormalTok{(}\AttributeTok{values =}\NormalTok{ cbPalette[}\FunctionTok{c}\NormalTok{(}\DecValTok{3}\NormalTok{,}\DecValTok{2}\NormalTok{, }\DecValTok{4}\NormalTok{)])}

\NormalTok{firstbloom\_comp\_diat\_plot}
\end{Highlighting}
\end{Shaded}

\begin{figure}
\centering
\includegraphics{okokyst_bloomtiming_files/figure-latex/unnamed-chunk-10-1.pdf}
\caption{First registration of the year of cell carbon above `bloom
level' for each species/taxon within the diatom genera Chaetoceros,
Pseudo-nitzschia and Skeletonema.}
\end{figure}

Conclusions:\\
- Blooming of members of these specific genera starts generally earlier
in the south.\\
- Some temporal separation between genera.

Cerataulina

\begin{Shaded}
\begin{Highlighting}[]
\NormalTok{level1 }\OtherTok{\textless{}{-}} \StringTok{"Genus"}
\NormalTok{level2 }\OtherTok{\textless{}{-}} \StringTok{"Genus"}


\NormalTok{firstbloom\_cerataulina }\OtherTok{\textless{}{-}}\NormalTok{ cellcarbon\_tax }\SpecialCharTok{\%\textgreater{}\%} \FunctionTok{mutate}\NormalTok{(}\AttributeTok{bloom =} \FunctionTok{ifelse}\NormalTok{(Verdi }\SpecialCharTok{\textgreater{}}\NormalTok{ bloomlevel, }\DecValTok{1}\NormalTok{, }\DecValTok{0}\NormalTok{)) }\SpecialCharTok{\%\textgreater{}\%} 
  \FunctionTok{filter}\NormalTok{(.data[[level1]] }\SpecialCharTok{\%in\%} \FunctionTok{c}\NormalTok{(}\StringTok{"Cerataulina"}\NormalTok{)) }\SpecialCharTok{\%\textgreater{}\%} \FunctionTok{filter}\NormalTok{(bloom }\SpecialCharTok{==} \DecValTok{1}\NormalTok{) }\SpecialCharTok{\%\textgreater{}\%} 
  \FunctionTok{group\_by}\NormalTok{(.data[[level2]], name, Year, station\_code, StationName\_new) }\SpecialCharTok{\%\textgreater{}\%} \FunctionTok{summarise\_at}\NormalTok{(}\FunctionTok{vars}\NormalTok{(doy), min) }\SpecialCharTok{\%\textgreater{}\%}  \CommentTok{\# add name when level2 != name}
  \FunctionTok{mutate}\NormalTok{(}\AttributeTok{Station\_code =} \FunctionTok{factor}\NormalTok{(station\_code, }\AttributeTok{levels =} \FunctionTok{unique}\NormalTok{(cellcarbon\_tax}\SpecialCharTok{$}\NormalTok{station\_code), }\AttributeTok{ordered =}\NormalTok{ T))}
\end{Highlighting}
\end{Shaded}

\begin{Shaded}
\begin{Highlighting}[]
\NormalTok{firstbloom\_cerataulina\_plot }\OtherTok{\textless{}{-}} \FunctionTok{ggplot}\NormalTok{(firstbloom\_cerataulina, }\FunctionTok{aes}\NormalTok{(}\AttributeTok{x =}\NormalTok{ Station\_code, }\AttributeTok{y =}\NormalTok{ doy, }\AttributeTok{color =}\NormalTok{ .data[[level2]], }\AttributeTok{shape =}\NormalTok{ Year, }
                       \AttributeTok{text =} \FunctionTok{sprintf}\NormalTok{(}\StringTok{"Taxon: \%s\textless{}br\textgreater{}Station: \%s\textless{}br\textgreater{} Doy: \%s"}\NormalTok{, .data[[level2]], StationName\_new, doy))) }\SpecialCharTok{+}
  \FunctionTok{geom\_jitter}\NormalTok{(}\AttributeTok{height =} \FloatTok{0.2}\NormalTok{, }\AttributeTok{width =} \FloatTok{0.2}\NormalTok{)}\SpecialCharTok{+}
  \FunctionTok{coord\_flip}\NormalTok{()}\SpecialCharTok{+}
  \FunctionTok{scale\_colour\_manual}\NormalTok{(}\AttributeTok{values =}\NormalTok{ cbPalette[}\FunctionTok{c}\NormalTok{(}\DecValTok{3}\NormalTok{,}\DecValTok{2}\NormalTok{, }\DecValTok{4}\NormalTok{)])}

\NormalTok{firstbloom\_cerataulina\_plot}
\end{Highlighting}
\end{Shaded}

\includegraphics{okokyst_bloomtiming_files/figure-latex/unnamed-chunk-12-1.pdf}
Conclusion:\\
- Strong north-south gradient in timing of first bloom of Cerataulina.
(BT133 and 132 did not have lat and long, are probably located further
south.)

\hypertarget{further-work}{%
\subsubsection{Further work:}\label{further-work}}

-Statistical tests: Co-occurrence/exclusion analyses? Other suggestions
are also welcome. -Separate the stations into Fjord, Sheltered,
Sheltered - fw-influenced, Exposed (c.~f.~Lømsland \& Johnsen), to see
if the patterns become clearer -Compare more genera, and species or size
categories within genera. I am conferring the Lømsland \& Johnsen paper
to find the overall most abundant genera in each region, but others
could also be of interest.

\end{document}
